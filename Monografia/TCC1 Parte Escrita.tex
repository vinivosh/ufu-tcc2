%!TEX TS-program = xelatex
%!BIB program = bibtex
%!LW recipe=xelatex-bibtex-xelatex-xelatex

%% abtex2-modelo-trabalho-academico.tex, v<VERSION> laurocesar
%% Copyright 2012-<COPYRIGHT_YEAR> by abnTeX2 group at http://www.abntex.net.br/ 
%%
%% This work may be distributed and/or modified under the
%% conditions of the LaTeX Project Public License, either version 1.3
%% of this license or (at your option) any later version.
%% The latest version of this license is in
%%   http://www.latex-project.org/lppl.txt
%% and version 1.3 or later is part of all distributions of LaTeX
%% version 2005/12/01 or later.
%%
%% This work has the LPPL maintenance status `maintained'.
%% 
%% The Current Maintainer of this work is the abnTeX2 team, led
%% by Lauro César Araujo. Further information are available on 
%% http://www.abntex.net.br/
%%
%% This work consists of the files abntex2-modelo-trabalho-academico.tex,
%% abntex2-modelo-include-comandos and abntex2-modelo-references.bib
%%

% ------------------------------------------------------------------------
% ------------------------------------------------------------------------
% abnTeX2: Modelo de Trabalho Academico (tese de doutorado, dissertacao de
% mestrado e trabalhos monograficos em geral) em conformidade com 
% ABNT NBR 14724:2011: Informacao e documentacao - Trabalhos academicos -
% Apresentacao
% ------------------------------------------------------------------------
% ------------------------------------------------------------------------

\documentclass[
	% -- opções da classe memoir --
	12pt,				% tamanho da fonte
	openright,			% capítulos começam em pág ímpar (insere página vazia caso preciso)
	twoside,			% para impressão em recto e verso. Oposto a oneside
	a4paper,			% tamanho do papel. 
	% -- opções da classe abntex2 --
	%chapter=TITLE,		% títulos de capítulos convertidos em letras maiúsculas
	%section=TITLE,		% títulos de seções convertidos em letras maiúsculas
	%subsection=TITLE,	% títulos de subseções convertidos em letras maiúsculas
	%subsubsection=TITLE,% títulos de subsubseções convertidos em letras maiúsculas
	% -- opções do pacote babel --
	english,			% idioma adicional para hifenização
	french,				% idioma adicional para hifenização
	spanish,			% idioma adicional para hifenização
	brazil				% o último idioma é o principal do documento
	]{abntex2}

% ---
% Pacotes básicos 
% ---
\usepackage{lmodern}			% Usa a fonte Latin Modern			
\usepackage[T1]{fontenc}		% Selecao de codigos de fonte.
\usepackage[utf8]{inputenc}		% Codificacao do documento (conversão automática dos acentos)
\usepackage{indentfirst}		% Indenta o primeiro parágrafo de cada seção.
\usepackage{color}				% Controle das cores
\usepackage{graphicx}			% Inclusão de gráficos
\graphicspath{ {./Imagens/} }
\usepackage{microtype} 			% para melhorias de justificação
% ---
		
% ---
% Pacotes adicionais, usados apenas no âmbito do Modelo Canônico do abnteX2
% ---
\usepackage{lipsum}				% para geração de dummy text
% ---

% ---
% Pacotes de citações
% ---
\usepackage[brazilian,hyperpageref]{backref}	 % Paginas com as citações na bibl
\usepackage[alf]{abntex2cite}	% Citações padrão ABNT

% ---
% Pacotes adicionados por mim (além dos do modelo!)
% ---
\usepackage[autostyle=true]{csquotes}   %Pacote de quotations (aspas)
% \usepackage[table]{xcolor}
% ---

% --- 
% CONFIGURAÇÕES DE PACOTES
% --- 

% ---
% Configurações do pacote backref
% Usado sem a opção hyperpageref de backref
\renewcommand{\backrefpagesname}{Citado na(s) página(s):~}
% Texto padrão antes do número das páginas
\renewcommand{\backref}{}
% Define os textos da citação
\renewcommand*{\backrefalt}[4]{
	\ifcase #1 %
		Nenhuma citação no texto.%
	\or
		Citado na página #2.%
	\else
		Citado #1 vezes nas páginas #2.%
	\fi}%
% ---

% ---
% Informações de dados para CAPA e FOLHA DE ROSTO
% ---
\titulo{Paralelização de Algoritmos Notórios de\\ Clustering em GPUs NVIDIA}
\autor{Vinícius Henrique Almeida Praxedes}
\local{Brasil}
\data{2019, Novembro}
\orientador{Daniel Duarte Abdala}
%\coorientador{Equipe \abnTeX}
\instituicao{%
  Universidade Federal de Uberlândia -- UFU
  \par
  Faculdade de Computação
  \par
  Programa de Graduação}
\tipotrabalho{Monografia}
% O preambulo deve conter o tipo do trabalho, o objetivo, 
% o nome da instituição e a área de concentração 
\preambulo{Monografia acadêmica apresentada à Faculdade de Computação da Universidade Federal de Uberlândia como parte dos requisitos necessários à obtenção do diploma do curso de Ciência da Computação}
% ---


% ---
% Configurações de aparência do PDF final

% alterando o aspecto da cor azul
\definecolor{blue}{RGB}{41,5,195}

% informações do PDF
\makeatletter
\hypersetup{
     	%pagebackref=true,
		pdftitle={\@title}, 
		pdfauthor={\@author},
    	pdfsubject={\imprimirpreambulo},
	    pdfcreator={LaTeX with abnTeX2},
		pdfkeywords={abnt}{latex}{abntex}{abntex2}{trabalho acadêmico}, 
		colorlinks=true,       		% false: boxed links; true: colored links
    	linkcolor=blue,          	% color of internal links
    	citecolor=blue,        		% color of links to bibliography
    	filecolor=magenta,      		% color of file links
		urlcolor=blue,
		bookmarksdepth=4
}
\makeatother
% --- 

% ---
% Posiciona figuras e tabelas no topo da página quando adicionadas sozinhas
% em um página em branco. Ver https://github.com/abntex/abntex2/issues/170
\makeatletter
\setlength{\@fptop}{5pt} % Set distance from top of page to first float
\makeatother
% ---

% ---
% Possibilita criação de Quadros e Lista de quadros.
% Ver https://github.com/abntex/abntex2/issues/176
%
\newcommand{\quadroname}{Quadro}
\newcommand{\listofquadrosname}{Lista de quadros}

\newfloat[chapter]{quadro}{loq}{\quadroname}
\newlistof{listofquadros}{loq}{\listofquadrosname}
\newlistentry{quadro}{loq}{0}

% configurações para atender às regras da ABNT
\setfloatadjustment{quadro}{\centering}
\counterwithout{quadro}{chapter}
\renewcommand{\cftquadroname}{\quadroname\space} 
\renewcommand*{\cftquadroaftersnum}{\hfill--\hfill}

\setfloatlocations{quadro}{hbtp} % Ver https://github.com/abntex/abntex2/issues/176
% ---

% --- 
% Espaçamentos entre linhas e parágrafos 
% --- 

% O tamanho do parágrafo é dado por:
\setlength{\parindent}{1.3cm}

% Controle do espaçamento entre um parágrafo e outro:
\setlength{\parskip}{0.2cm}  % tente também \onelineskip

% ---
% compila o indice
% ---
\makeindex
% ---

% ----
% Variáveis e definições
\def\qntAlgrtm{quatro}
\def\qntAlgrtmNaoExtenso{4}
% ----

% ----
% Início do documento
% ----
\begin{document}

% Seleciona o idioma do documento (conforme pacotes do babel)
%\selectlanguage{english}
\selectlanguage{brazil}

% Retira espaço extra obsoleto entre as frases.
\frenchspacing 

% ----------------------------------------------------------
% ELEMENTOS PRÉ-TEXTUAIS
% ----------------------------------------------------------
% \pretextual

% ---
% Capa
% ---
\imprimircapa
% ---

% ---
% Folha de rosto
% (o * indica que haverá a ficha bibliográfica)
% ---
\imprimirfolhaderosto*
% ---

% ---
% Inserir a ficha bibliografica
% ---

% Isto é um exemplo de Ficha Catalográfica, ou ``Dados internacionais de
% catalogação-na-publicação''. Você pode utilizar este modelo como referência. 
% Porém, provavelmente a biblioteca da sua universidade lhe fornecerá um PDF
% com a ficha catalográfica definitiva após a defesa do trabalho. Quando estiver
% com o documento, salve-o como PDF no diretório do seu projeto e substitua todo
% o conteúdo de implementação deste arquivo pelo comando abaixo:
%
% \begin{fichacatalografica}
%     \includepdf{fig_ficha_catalografica.pdf}
% \end{fichacatalografica}

\begin{fichacatalografica}
	\sffamily
	\vspace*{\fill}					% Posição vertical
	\begin{center}					% Minipage Centralizado
	\fbox{\begin{minipage}[c][8cm]{13.5cm}		% Largura
	\small
	\imprimirautor
	%Sobrenome, Nome do autor
	
	\hspace{0.5cm} \imprimirtitulo  / \imprimirautor. --
	\imprimirlocal, \imprimirdata-
	
	\hspace{0.5cm} \thelastpage p. : il. (algumas color.) ; 30 cm.\\
	
	\hspace{0.5cm} \imprimirorientadorRotulo~\imprimirorientador\\
	
	\hspace{0.5cm}
	\parbox[t]{\textwidth}{\imprimirtipotrabalho~--~\imprimirinstituicao,
	\imprimirdata.}\\
	
	\hspace{0.5cm}
		1. Palavra-chave1.
		2. Palavra-chave2.
		2. Palavra-chave3.
		I. Orientador.
		II. Universidade Federal de Uberlândia.
		III. Faculdade de Computação.
		IV. Título 			
	\end{minipage}}
	\end{center}
\end{fichacatalografica}
% ---

% ---
% Inserir errata
% ---
% \begin{errata}
% Elemento opcional da \citeonline[4.2.1.2]{NBR14724:2011}. Exemplo:

% \vspace{\onelineskip}

% FERRIGNO, C. R. A. \textbf{Tratamento de neoplasias ósseas apendiculares com
% reimplantação de enxerto ósseo autólogo autoclavado associado ao plasma
% rico em plaquetas}: estudo crítico na cirurgia de preservação de membro em
% cães. 2011. 128 f. Tese (Livre-Docência) - Faculdade de Medicina Veterinária e
% Zootecnia, Universidade de São Paulo, São Paulo, 2011.

% \begin{table}[htb]
% \center
% \footnotesize
% \begin{tabular}{|p{1.4cm}|p{1cm}|p{3cm}|p{3cm}|}
%   \hline
%   \textbf{Folha} & \textbf{Linha}  & \textbf{Onde se lê}  & \textbf{Leia-se}  \\
%     \hline
%     1 & 10 & auto-conclavo & autoconclavo\\
%   \hline
% \end{tabular}
% \end{table}

% \end{errata}
% ---

% ---
% Inserir folha de aprovação
% ---

% Isto é um exemplo de Folha de aprovação, elemento obrigatório da NBR
% 14724/2011 (seção 4.2.1.3). Você pode utilizar este modelo até a aprovação
% do trabalho. Após isso, substitua todo o conteúdo deste arquivo por uma
% imagem da página assinada pela banca com o comando abaixo:
%
% \begin{folhadeaprovacao}
% \includepdf{folhadeaprovacao_final.pdf}
% \end{folhadeaprovacao}
%
\begin{folhadeaprovacao}

  \begin{center}
    {\ABNTEXchapterfont\large\imprimirautor}

    \vspace*{\fill}\vspace*{\fill}
    \begin{center}
      \ABNTEXchapterfont\bfseries\Large\imprimirtitulo
    \end{center}
    \vspace*{\fill}
    
    \hspace{.45\textwidth}
    \begin{minipage}{.5\textwidth}
        \imprimirpreambulo
    \end{minipage}%
    \vspace*{\fill}
   \end{center}
        
   Trabalho aprovado. \imprimirlocal, 24 de novembro de 2012:

   \assinatura{\textbf{\imprimirorientador} \\ Orientador} 
   \assinatura{\textbf{Professor} \\ Convidado 1}
   \assinatura{\textbf{Professor} \\ Convidado 2}
   %\assinatura{\textbf{Professor} \\ Convidado 3}
   %\assinatura{\textbf{Professor} \\ Convidado 4}
      
   \begin{center}
    \vspace*{0.5cm}
    {\large\imprimirlocal}
    \par
    {\large\imprimirdata}
    \vspace*{1cm}
  \end{center}
  
\end{folhadeaprovacao}
% ---

% % ---
% % Dedicatória
% % ---
% \begin{dedicatoria}
%   \vspace*{\fill}
%   \centering
%   \noindent
%   \textit{ Este trabalho é dedicado às crianças adultas que,\\
%   quando pequenas, sonharam em se tornar cientistas.} \vspace*{\fill}
% \end{dedicatoria}
% % ---

% % ---
% % Agradecimentos
% % ---
% \begin{agradecimentos}
% Os agradecimentos principais são direcionados à Gerald Weber, Miguel Frasson,
% Leslie H. Watter, Bruno Parente Lima, Flávio de Vasconcellos Corrêa, Otavio Real
% Salvador, Renato Machnievscz\footnote{Os nomes dos integrantes do primeiro
% projeto abn\TeX\ foram extraídos de
% \url{http://codigolivre.org.br/projects/abntex/}} e todos aqueles que
% contribuíram para que a produção de trabalhos acadêmicos conforme
% as normas ABNT com \LaTeX\ fosse possível.

% Agradecimentos especiais são direcionados ao Centro de Pesquisa em Arquitetura
% da Informação\footnote{\url{http://www.cpai.unb.br/}} da Universidade de
% Brasília (CPAI), ao grupo de usuários
% \emph{latex-br}\footnote{\url{http://groups.google.com/group/latex-br}} e aos
% novos voluntários do grupo
% \emph{\abnTeX}\footnote{\url{http://groups.google.com/group/abntex2} e
% \url{http://www.abntex.net.br/}}~que contribuíram e que ainda
% contribuirão para a evolução do \abnTeX.

% \end{agradecimentos}
% % ---

% % ---
% % Epígrafe
% % ---
% \begin{epigrafe}
%     \vspace*{\fill}
% 	\begin{flushright}
% 		\textit{``Não vos amoldeis às estruturas deste mundo, \\
% 		mas transformai-vos pela renovação da mente, \\
% 		a fim de distinguir qual é a vontade de Deus: \\
% 		o que é bom, o que Lhe é agradável, o que é perfeito.\\
% 		(Bíblia Sagrada, Romanos 12, 2)}
% 	\end{flushright}
% \end{epigrafe}
% % ---

% % ---
% % RESUMOS
% % ---

% % resumo em português
% \setlength{\absparsep}{18pt} % ajusta o espaçamento dos parágrafos do resumo
% \begin{resumo}
%  Segundo a \citeonline[3.1-3.2]{NBR6028:2003}, o resumo deve ressaltar o
%  objetivo, o método, os resultados e as conclusões do documento. A ordem e a extensão
%  destes itens dependem do tipo de resumo (informativo ou indicativo) e do
%  tratamento que cada item recebe no documento original. O resumo deve ser
%  precedido da referência do documento, com exceção do resumo inserido no
%  próprio documento. (\ldots) As palavras-chave devem figurar logo abaixo do
%  resumo, antecedidas da expressão Palavras-chave:, separadas entre si por
%  ponto e finalizadas também por ponto.

%  \textbf{Palavras-chave}: latex. abntex. editoração de texto.
% \end{resumo}

% % resumo em inglês
% \begin{resumo}[Abstract]
%  \begin{otherlanguage*}{english}
%   This is the english abstract.

%   \vspace{\onelineskip}
 
%   \noindent 
%   \textbf{Keywords}: latex. abntex. text editoration.
%  \end{otherlanguage*}
% \end{resumo}

% % resumo em francês 
% \begin{resumo}[Résumé]
%  \begin{otherlanguage*}{french}
%     Il s'agit d'un résumé en français.
 
%   \textbf{Mots-clés}: latex. abntex. publication de textes.
%  \end{otherlanguage*}
% \end{resumo}

% % resumo em espanhol
% \begin{resumo}[Resumen]
%  \begin{otherlanguage*}{spanish}
%   Este es el resumen en español.
  
%   \textbf{Palabras clave}: latex. abntex. publicación de textos.
%  \end{otherlanguage*}
% \end{resumo}
% % ---

% % ---
% % inserir lista de ilustrações
% % ---
% \pdfbookmark[0]{\listfigurename}{lof}
% \listoffigures*
% \cleardoublepage
% % ---

% % ---
% % inserir lista de quadros
% % ---
% \pdfbookmark[0]{\listofquadrosname}{loq}
% \listofquadros*
% \cleardoublepage
% % ---

% % ---
% % inserir lista de tabelas
% % ---
% \pdfbookmark[0]{\listtablename}{lot}
% \listoftables*
% \cleardoublepage
% % ---

% % ---
% % inserir lista de abreviaturas e siglas
% % ---
% \begin{siglas}
%   \item[ABNT] Associação Brasileira de Normas Técnicas
%   \item[abnTeX] ABsurdas Normas para TeX
% \end{siglas}
% % ---

% % ---
% % inserir lista de símbolos
% % ---
% \begin{simbolos}
%   \item[$ \Gamma $] Letra grega Gama
%   \item[$ \Lambda $] Lambda
%   \item[$ \zeta $] Letra grega minúscula zeta
%   \item[$ \in $] Pertence
% \end{simbolos}
% % ---

% ---
% inserir o sumario
% ---
\pdfbookmark[0]{\contentsname}{toc}
\tableofcontents*
\cleardoublepage
% ---



% ----------------------------------------------------------
% ELEMENTOS TEXTUAIS
% ----------------------------------------------------------
\textual

% ----------------------------------------------------------
% Capítulo - Introdução
% ----------------------------------------------------------
\chapter{Introdução}
% ----------------------------------------------------------

A busca pelo menor tempo de execução é uma diretriz ubíqua na computação. Desde os primórdios da área buscamos algoritmos e procedimentos que, dados os mesmos parâmetros de entrada, executem a mesma tarefa na menor quantidade de tempo possível. Outros recursos como espaço de memória utilizado, eficiência energética ou uso da rede em muitos cenários são mais importantes que o tempo de execução, mas ainda sim a mesma continua sendo um dos mais estudados parâmetros para categorização e avaliação de algoritmos e procedimentos na computação. De fato o tempo de execução -- em ciclos, ou passos, de processamento -- é a métrica utilizada na análise de uma das maiores incógnitas da computação, o problema P versus NP.

Um grande avanço na quantidade de poder de processamento dos computadores e, portanto, diminuição do tempo de execução de algoritmos, foi a paralelização dos processadores. A habilidade de poder executar duas ou mais ações simultaneamente possibilitou muitos ganhos palpáveis na velocidade de execução de algoritmos e procedimentos, porém introduziu uma necessidade de mudança na forma de se pensar em resoluções de problemas computacionalmente: paralelizar um algoritmo antes serial não é uma tarefa trivial, e requer cuidados especiais com concorrência no acesso a recursos da máquina, dependência de dados e cálculos, sincronização, entre outros dilemas.

Um dos componentes que mais utilizam da paralelização num computador moderno são as GPUs, unidades de processamento gráfico, ou placas de vídeo, que são basicamente processadores especializados em operações vetoriais, altamente paralelizadas, usualmente dedicados a operações gráficas, e com sua própria memória dedicada, a VRAM. Enquanto processadores de uso geral, CPUs, costumam ter no máximo dezenas de núcleos para processamento paralelo, GPUs possuem milhares ou até milhões de núcleos para operações vetoriais.

No entanto, cada vez mais está sendo descoberto e aproveitado o potencial de uso das GPUs em atividades não apenas voltadas para renderização, interfaces e outras operações gráficas, mas para a computação de propósito geral. Diversos algoritmos modernos e antigos beneficiam-se imensamente da paralelização enorme proporcionada por GPUs, e com ferramentas como a biblioteca e linguagem CUDA criada pela NVIDIA, está cada vez mais fácil implementar o uso de placas de vídeo em conjunto com CPUs convencionais nos mais variados algoritmos.

Nem todo algoritmo pode ser paralelizado, no entanto. Existem procedimentos e algoritmos que são inerentemente seriais, como o cálculo do $n$-ésimo número da sequência de \textit{Fibonacci}, que requer que dois números prévios da sequência tenham sido calculados para obtermos o atual -- salvo, é claro, alguma descoberta teórica matemática do comportamento da sequência que nos permitisse uma nova maneira de calcular o $n$-ésimo elemento sem essa necessidade.

É importante entender também que nenhum algoritmo é paralelizável por completo. Sempre existem partes de algoritmos que devem ser executadas serialmente para funcionar corretamente. O ganho máximo da paralelização de um algoritmo qualquer é uma função da parcela do seu código que pode ser paralelizada, e é definida pela Lei de Amdahl \cite{Amdahl-Law}: $\frac{1}{1-p}$, onde $p$ é a razão entre código paralelizável e código não-paralelizável.

E é a paralelização de uma classe de algoritmos em especial que é o foco desta pesquisa: algoritmos de \textit{clustering}, ou clusterização. Tais algoritmos, de forma sucinta, agrupam objetos de maneira que os objetos no mesmo grupo, ou \textit{cluster}, sejam mais parecidos entre si, de acordo com alguma métrica, do que com objetos de outros grupos. A análise de clusters é essencial em diversas áreas da computação, como mineração de dados, aprendizado de máquina, compressão de dados, entre outras.

% !!!!!!!!!!!!!!!!!!!!!!!!!!!!
% Checar se essa citação do site da NVIDIA está correta! Esse bagulho no Bib usou o modelo "misc", não sei se está correto. O Mendeley gerou assim, então talvez esteja...
% !!!!!!!!!!!!!!!!!!!!!!!!!!!!

A hipótese principal é a de que algoritmos de clustering, em geral, são altamente paralelizáveis, e apresentam um ganho considerável de desempenho (menor tempo de execução) quando implementados para utilizar o poder de paralelismo vetorial de placas de vídeo NVIDIA, através da linguagem CUDA \cite{CUDAZone}. Mais que isso, através de uma análise sistemática de estudos prévios e implementações de tais algoritmos em CUDA, visa-se generalizar o processo de paralelização destes. Isto é, identificar quais partes são necessariamente seriais, quais são paralelizáveis, e que sequência de passos gerais deve ser seguida para se conseguir paralelizar com sucesso e ganho significativo de desempenho um algoritmo de clustering qualquer.

% !!!!!!!!!!!!!!!!!!!!!!!!!!!!
% Tenho que substituir essas referências aqui por referências de versões seriais de cada algoritmo. Usar as referências das versões aceleradas em GPUs mais adiante, ao invés de aqui parece fazer bem mais sentido
% !!!!!!!!!!!!!!!!!!!!!!!!!!!!

O foco de pesquisa são \qntAlgrtm{} algoritmos de clustering especialmente notórios: \textit{K-means} \cite{GPU-accelerated-K-Means}, \textit{Clustering Hierárquico}, \textit{DBSCAN} \cite{G-DBSCAN}, ou \textit{Clusterização Espacial Baseada em Densidade de Aplicações com Ruído}, e \textit{Random Forests} -- que mesmo não sendo exclusivamente um algoritmo de clusterização, pode ser utilizado justamente para tal. Implementações e estudos realizados sobre estes serão analisados, e implementações novas terão desempenho comparado com estas.

% ----

\section{Objetivos}

\subsection{Objetivo Geral}

Esse trabalho tem como objetivo principal testar a validez de sua hipótese (discutida mais à fundo na seção 1.2) de que algoritmos de clusterização são intrinsecamente paralelizáveis, e que o ganho de velocidade ao serem paralelizados é altamente significativo.

Além disso, deseja-se compilar aqui um vasto conhecimento de como paralelizar esses algoritmos em geral, analisando principalmente os \qntAlgrtm{} aqui estudados a fundo (k-means, clustering hierárquico, DBSCAN e Random Forests) e usando este aprendizado para criar um passo-a-passo genérico de como realizar tal modificação de código em um algoritmo de clusterização qualquer.

\subsection{Objetivos Específicos}

Para atingir o objetivo geral, será necessário completar diversos objetivos menores, ou \textit{milestones}, antes, criando um caminho de pesquisa a ser seguido --- não necessariamente na ordem apresentada. São estes:

\begin{itemize}
    \item Pesquisar extensamente a bibliografia da área, realizando assim um levantamento do estado da arte de algoritmos paralelos de clusterização;
    \item Estudar implementações já realizadas dos \qntAlgrtm{} algoritmos aqui estudados, a fim de adquirir conhecimento de como a paralelização em CUDA deve ser realizada;
    % !!!!!!!!!!!!!!!!!!!!!!!!!!!!
    % Tenho que aprofundar nesse tópico, explicando qual algoritmo de clustering "novo" será paralelizado afinal, quando eu escolher um de fato!
    % !!!!!!!!!!!!!!!!!!!!!!!!!!!!
    \item Paralelizar um algoritmo de clustering \enquote{novo}, isto é, nunca antes paralelizado e exibido em trabalho científico, a fim de solidificar o conhecimento e prática de programação em CUDA;
    \item Quantificar o ganho de desempenho das implementações paralelas, realizando diversos experimentos de \textit{speedup};
    \item Comparar código monolítico (sem paralelização) com código paralelo dos \qntAlgrtm{} algoritmos analisados, extraindo assim um conhecimento de como paralelizar um algoritmo de clusterização genérico;
\end{itemize}

% ----

\section{Hipótese}

A hipótese que esta pesquisa procura testar é a de que algoritmos de clusterização em geral são inerentemente vetoriais e, consequentemente, se beneficiariam significativamente de arquiteturas de processamento vetoriais, como uma unidade de processamento gráfico, ou GPU.

Um problema ser vetorial diz respeito ao escopo de itens de dados relevantes ao problema. Grande parte dos problemas da computação são escalares, o que significa que eles lidam com itens de dados unitários, como por exemplo \textit{integers} ou \textit{floats}, um de cada vez. Já um problema vetorial lida com itens de dados que são conjuntos unidimensionais, chamados vetores, que são formados por vários itens unitários de dados agrupados.

Um algoritmo que tente resolver um problema vetorial terá desempenho maior quando executado num processador vetorial, isto é, um processador que possui um conjunto de instruções capaz de manipular vetores. Apesar de um algoritmo de um problema vetorial ainda puder ser implementado e executado com sucesso num processador escalar, o desempenho será menor pois os dados vetoriais do problema terão que ter seus elementos processados um a um pelo processador, já que ele não trabalha com vetores propriamente ditos em seu conjunto de instruções.

Grande parte do ganho de desempenho supracitado vem do paralelismo proporcionado pelos processadores vetoriais, como GPUs, ao manipular conjuntos maiores de dados de uma só vez, e em vários núcleos simultaneamente, além de economizar traduções de endereço de memória e operações de obtenção (\textit{fetch}) e decodificação (\textit{decode}) de instruções, pois haverá um número muito menor de instruções e endereços de memória quando os dados estão agrupados em vetores, que podem ser manipulados e usados em operações como se fossem apenas um item de dados.

Este trabalho, então, visa demonstrar que algoritmos de clusterização, em geral, são intrinsecamente vetoriais. Isto é, qualquer algoritmo de clustering concebível será de natureza vetorial, pois estes analisam dados e tentam agrupá-los de a acordo com algum grau de semelhança entre eles, análise essa que pode ser feita com conjuntos dos itens de dados (vetores), ao invés de individualmente. Logo, qualquer algoritmo de clusterização teria uma parcela de seu código que seria paralelizável, e assim ganhariam desempenho significativo com uma execução numa GPU.

% ----

\section{Justificativa}

A pesquisa feita aqui pode ser de grande utilidade para a área de análise de clusters, e impulsionar a implementação de mais algoritmos paralelos de clusterização.

Com a compilação de conhecimento realizada aqui a intenção é facilitar pesquisas posteriores na área de paralelização de algoritmos de clustering e motivar novas implementações paralelas de outros algoritmos dessa classe com os experimentos de ganho de desempenho, ilustrando o quão importante é o uso de processadores vetoriais como GPUs para tornar o uso de alguns destes algoritmos prático.

Além disso, a apresentação nesse estudo de um procedimento genérico para paralelizar qualquer algoritmo de clustering será de extrema utilidade para qualquer desenvolvedor ou pesquisador que desejar implementar uma versão acelerada em GPU de um algoritmo de clusterização, mesmo este sendo totalmente novo. No mínimo servirá de ponto de partida para o entendimento e aprendizado de como realizar tal modificação no código do algoritmo, e renderá uma implementação que serve de base para estudos e otimizações, até se atingir uma implementação digna para uso prático.

% ----------------------------------------------------------
% Capítulo - Fundamentação Teórica
% ----------------------------------------------------------

\chapter{Fundamentação Teórica}

Para compreender a pesquisa científica aqui realizada, é necessário primeiro entender o que são algoritmos de clusterização, tanto de maneira geral quanto específica, explorando os fundamentos e funcionamento dos \qntAlgrtm{} algoritmos pesquisados. Veremos que a complexidade de tempo destes algoritmos tendem a ser inconvenientemente altas ($O(n\cdot\log{n})$, $O(n^2)$ ou até $O(n^3)$ sendo complexidades comuns) e por isso qualquer ganho de velocidade significativo obtido valerá muito a pena.

Também é imprescindível explorar o funcionamento dos processadores vetoriais --- sendo as \textit{Unidades de Processamento Gráfico} (GPUs) o principal exemplo destes e sobre o qual essa pesquisa irá focar --- e entender por quê usá-los para paralelizar algoritmos de clusterização proporcionará, em tese, um ganho de velocidade expressivo na execução destes, abrandando o peso de suas complexidades de tempo. Além de tudo isso, será apresentada brevemente a arquitetura que será utilizada para paralelizar os algoritmos estudados: a plataforma e modelo de programação CUDA, da NVIDIA, que permitirá extrair o poder de paralelização das placas de vídeo NVIDIA.

% ----

\section{Clusterização}

A clusterização, ou clustering, é a tarefa de agrupar um conjunto de elementos de modo que cada elemento de um grupo se \enquote{pareça} mais com outros elementos do grupo (cluster) que pertence do que com elementos dos grupos que não pertence --- para algum significado bem definido de semelhança. É um processo muito comum e virtualmente necessário nas áreas de mineração de dados, análise estatística, análise de imagem, aprendizado de máquina, reconhecimento de padrões, e muitas outras.

O significado de um cluster não pode ser bem definido e vai depender do conjunto de dados a ser analisado e a forma que os resultados obtidos serão utilizados --- de fato, esse é o principal motivo pelo qual tantos algoritmos diferentes de clustering existem \cite{SoManyClustAlg}. O comum em todas definições usadas é que um cluster é um conjunto de objetos de dados. Esses objetos são representados num espaço de dimensões iguais ao número de dados relevantes para cada objeto, e os algoritmos tentam criar grupos (clusters) nesse espaço que agrupem os objetos de uma maneira significativa, ou útil, para o estudo sendo feito e a definição de cluster sendo utilizada.

% !!!!!!!!!!!!!!!!!!!!!!!!!!!!
% Talvez seria melhor colocar esse parágrafo num formato de lista? Não sei...
% Também não sei se meu uso do negrito aqui é permitido formalmente.
% !!!!!!!!!!!!!!!!!!!!!!!!!!!!

Diversos modelos de cluster podem ser usados para definir o que é um cluster: \textbf{modelos de centroide}, onde cada cluster possui um centro e os objetos pertencerão aos clusters com centros mais próximos deles, dada uma definição de distância no espaço de dados; \textbf{modelos de densidade}, que definem clusters como regiões densas e conexas no espaço de dados, contrastando com regiões menos densas que separam os clusters; \textbf{modelos de conectividade}, que constroem clusters a partir de conexões de objetos por distância; \textbf{modelos de distribuição}, que utilizam de distribuições estatísticas, como a distribuição normal ou exponencial, para modelar agrupamentos dos objetos; entre dezenas de outros modelos. Entender o modelo de cluster utilizado é essencial para compreender um algoritmo de clusterização e as diferenças entre a multitude destes.

% !!!!!!!!!!!!!!!!!!!!!!!!!!!!
% Eu preciso realmente ficar colocando termos em inglês não traduzidos (intraduzíveis de maneira sã, na maioria das vezes) em itálico? Isso me parece bem brega.
% !!!!!!!!!!!!!!!!!!!!!!!!!!!!

O resultado, ou saída, de um algoritmo de clusterização é comumente um rotulamento dos objetos de dados passados na entrada, o que indicará a divisão em clusters feita por ele. Classificações podem ser feitas quanto à natureza da clusterização obtida pelos algoritmos: \textbf{\textit{hard clustering}}, onde cada objeto pertence ou não a um cluster; \textbf{\textit{fuzzy clustering}}, onde cada objeto pertence uma certa porcentagem a cada cluster, o que pode significar, por exemplo, a chance do objeto pertencer àquele cluster \cite{FuzzyClusteringSurvey}. E subclassificações mais granulares ainda podem ser definidas, como: \textbf{clusterização de particionamento estrito}, onde cada objeto pertence a exatamente um cluster; \textbf{clusterização de particionamento estrito com \textit{outliers}}, onde objetos pertencem a exatamente um cluster, ou nenhum cluster, assim sendo considerados \textit{outliers}.

% \subsection{K-Means}

% \subsection{Clusterização Hierárquica}

% \subsection{DBSCAN}

% \subsection{Random Forests}

% ----

% \section{Processadores Vetoriais}

% \subsection{A Arquitetura NVIDIA / CUDA}

% \subsection{Como Programar em CUDA: Um Overview}

% ----

% \chapter{Levantamento do Estado da Arte}

% ----------------------------------------------------------
% Capítulo - Metodologia de Desenvolvimento e Pesquisa
% ----------------------------------------------------------

\chapter{Metodologia de Desenvolvimento e Pesquisa}

A pesquisa realizada neste trabalho consiste de estudos e análises de trabalhos prévios, desenvolvimento de uma versão paralelizada de um algoritmo de clustering e experimentos sobre essa implementação. Pode-se dividir tal metodologia em um conjunto de etapas.

A primeira etapa consiste de uma extensa pesquisa bibliográfica. O intuito é levantar o estado da arte na área de algoritmos de clusterização acelerados em GPU usando a linguagem e livraria CUDA da NVIDIA. Entender quais algoritmos já foram implementados com sucesso em CUDA, e como foi feita tal implementação, além dos resultados (ganhos em desempenho) das mesmas. Essa etapa agregará conhecimento sobre como utilizar CUDA para acelerar algoritmos de clustering, além de dar uma ideia do tipo de ganho de desempenho esperado de uma paralelização média desse tipo de algoritmo.

A segunda etapa consiste da implementação de uma versão paralela \enquote{inédita} de algum algum algoritmo de clusterização, ou seja, paralelizar um algoritmo nunca antes paralelizado e exibido em alguma pesquisa. Usando o aprendizado adquirido na primeira etapa de pesquisa, um algoritmo será paralelizado em CUDA, e seus resultados comparados com os resultados da versão serial (rodando somente numa CPU) para garantir corretude. Os ganhos de velocidade da versão acelerada em CUDA será então comparada também com os ganhos obtidos nos trabalhos analisados na primeira etapa. Isso servirá como uma medição da efetividade da implementação feita.

A terceira etapa consiste da busca de um procedimento geral para paralelizar um algoritmo de clustering genérico. Ou seja, encontrar um passo-a-passo de modificações ao código de um algoritmo serial que, ao fim, o transforme numa versão acelerada usando CUDA desse mesmo algoritmo, ainda mantendo sua corretude e proporcionando algum ganho de desempenho.

A quarta etapa, por fim, se trata de diversos experimentos de ganho de velocidade, ou \textit{speedup}, do algoritmo que teve aqui sua versão acelerada em GPU implementada e apresentada. Esses experimentos irão dar uma ideia do quão significativo foi o ganho de desempenho ao paralelizar o algoritmo usando CUDA, se há um teto ou chão para tais ganhos, como o \textit{speedup} aumenta ou diminui com o aumento do dataset a ser analisado, assim como realizar uma análise de outros parâmetros importantes que não sejam velocidade, como uso de memória --- afinal, as VRAMs das placas de vídeo são comumente mais limitadas em tamanho que as RAMs utilizadas pelas CPUs.

% \chapter{Experimentos e Datasets}

% \chapter{Conclusões}

% ----------------------------------------------------------
% Capítulo - Cronograma
% ----------------------------------------------------------

\chapter{Cronograma}

A tabela abaixo representa o cronograma de pesquisa e desenvolvimento deste trabalho. Cada célula da mesma representa uma semana de tempo (sete dias corridos), considerando cada mês como tendo exatamente quatro semanas. O período coberto vai de dezembro de 2019 à julho de 2020, data estimada de término da conclusão desta monografia.

\begin{figure}[h]
    \centering
    \includegraphics[width=\textwidth]{Cronograma_2-0}
    \caption{Cronograma de pesquisa e desenvolvimento}
    % \label{fig:my_label}
\end{figure}

% ----------------------------------------------------------

% % ----------------------------------------------------------
% % PARTE
% % ----------------------------------------------------------
% %\part{Preparação da pesquisa}
% % ----------------------------------------------------------

% % ---
% % Capitulo com exemplos de comandos inseridos de arquivo externo 
% % ---
% \include{abntex2-modelo-include-comandos}
% % ---

% \chapter{Conteúdos específicos do modelo de trabalho acadêmico}\label{cap_trabalho_academico}

% \section{Quadros}

% Este modelo vem com o ambiente \texttt{quadro} e impressão de Lista de quadros 
% configurados por padrão. Verifique um exemplo de utilização:

% \begin{quadro}[htb]
% \caption{\label{quadro_exemplo}Exemplo de quadro}
% \begin{tabular}{|c|c|c|c|}
% 	\hline
% 	\textbf{Pessoa} & \textbf{Idade} & \textbf{Peso} & \textbf{Altura} \\ \hline
% 	Marcos & 26    & 68   & 178    \\ \hline
% 	Ivone  & 22    & 57   & 162    \\ \hline
% 	...    & ...   & ...  & ...    \\ \hline
% 	Sueli  & 40    & 65   & 153    \\ \hline
% \end{tabular}
% \fonte{Autor.}
% \end{quadro}

% Este parágrafo apresenta como referenciar o quadro no texto, requisito
% obrigatório da ABNT. 
% Primeira opção, utilizando \texttt{autoref}: Ver o \autoref{quadro_exemplo}. 
% Segunda opção, utilizando  \texttt{ref}: Ver o Quadro \ref{quadro_exemplo}.

% % ----------------------------------------------------------
% % PARTE
% % ----------------------------------------------------------
% \part{Referenciais teóricos}
% % ----------------------------------------------------------

% % ---
% % Capitulo de revisão de literatura
% % ---
% \chapter{Lorem ipsum dolor sit amet}
% % ---

% % ---
% \section{Aliquam vestibulum fringilla lorem}
% % ---

% \lipsum[1]

% \lipsum[2-3]

% % ----------------------------------------------------------
% % PARTE
% % ----------------------------------------------------------
% \part{Resultados}
% % ----------------------------------------------------------

% % ---
% % primeiro capitulo de Resultados
% % ---
% \chapter{Lectus lobortis condimentum}
% % ---

% % ---
% \section{Vestibulum ante ipsum primis in faucibus orci luctus et ultrices
% posuere cubilia Curae}
% % ---

% \lipsum[21-22]

% % ---
% % segundo capitulo de Resultados
% % ---
% \chapter{Nam sed tellus sit amet lectus urna ullamcorper tristique interdum
% elementum}
% % ---

% % ---
% \section{Pellentesque sit amet pede ac sem eleifend consectetuer}
% % ---

% \lipsum[24]

% % ----------------------------------------------------------
% % Finaliza a parte no bookmark do PDF
% % para que se inicie o bookmark na raiz
% % e adiciona espaço de parte no Sumário
% % ----------------------------------------------------------
% \phantompart

% % ---
% % Conclusão
% % ---
% \chapter{Conclusão}
% % ---

% \lipsum[31-33]

% ----------------------------------------------------------
% ELEMENTOS PÓS-TEXTUAIS
% ----------------------------------------------------------
\postextual
% ----------------------------------------------------------

% ----------------------------------------------------------
% Referências bibliográficas
% ----------------------------------------------------------
\bibliography{library}

% % ----------------------------------------------------------
% % Glossário
% % ----------------------------------------------------------
% %
% % Consulte o manual da classe abntex2 para orientações sobre o glossário.
% %
% %\glossary

% % ----------------------------------------------------------
% % Apêndices
% % ----------------------------------------------------------

% % ---
% % Inicia os apêndices
% % ---
% \begin{apendicesenv}

% % Imprime uma página indicando o início dos apêndices
% \partapendices

% % ----------------------------------------------------------
% \chapter{Quisque libero justo}
% % ----------------------------------------------------------

% \lipsum[50]

% % ----------------------------------------------------------
% \chapter{Nullam elementum urna vel imperdiet sodales elit ipsum pharetra ligula
% ac pretium ante justo a nulla curabitur tristique arcu eu metus}
% % ----------------------------------------------------------
% \lipsum[55-57]

% \end{apendicesenv}
% % ---


% % ----------------------------------------------------------
% % Anexos
% % ----------------------------------------------------------

% % ---
% % Inicia os anexos
% % ---
% \begin{anexosenv}

% % Imprime uma página indicando o início dos anexos
% \partanexos

% % ---
% \chapter{Morbi ultrices rutrum lorem.}
% % ---
% \lipsum[30]

% % ---
% \chapter{Cras non urna sed feugiat cum sociis natoque penatibus et magnis dis
% parturient montes nascetur ridiculus mus}
% % ---

% \lipsum[31]

% % ---
% \chapter{Fusce facilisis lacinia dui}
% % ---

% \lipsum[32]

% \end{anexosenv}

%---------------------------------------------------------------------
% INDICE REMISSIVO
%---------------------------------------------------------------------
\phantompart
\printindex
%---------------------------------------------------------------------

\end{document}
